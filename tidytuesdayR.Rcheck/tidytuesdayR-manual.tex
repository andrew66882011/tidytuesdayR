\nonstopmode{}
\documentclass[letterpaper]{book}
\usepackage[times,inconsolata,hyper]{Rd}
\usepackage{makeidx}
\usepackage[utf8]{inputenc} % @SET ENCODING@
% \usepackage{graphicx} % @USE GRAPHICX@
\makeindex{}
\begin{document}
\chapter*{}
\begin{center}
{\textbf{\huge Package `tidytuesdayR'}}
\par\bigskip{\large \today}
\end{center}
\begin{description}
\raggedright{}
\inputencoding{utf8}
\item[Type]\AsIs{Package}
\item[Title]\AsIs{Access the Weekly 'TidyTuesday' Project Dataset}
\item[Version]\AsIs{1.0.0}
\item[Description]\AsIs{'TidyTuesday' is a project by the 'R4DS Online Learning Community' in which they
post a weekly dataset onto post a weekly dataset in a public data repository
(<https://github.com/rfordatascience/tidytuesday>) for people to
analyze and visualize. This package provides the tools to easily download this data and the
description of the source.}
\item[License]\AsIs{MIT + file LICENSE}
\item[URL]\AsIs{}\url{https://github.com/thebioengineer/tidytuesdayR}\AsIs{}
\item[BugReports]\AsIs{}\url{https://github.com/thebioengineer/tidytuesdayR/issues}\AsIs{}
\item[Encoding]\AsIs{UTF-8}
\item[LazyData]\AsIs{true}
\item[RoxygenNote]\AsIs{7.1.0}
\item[Depends]\AsIs{R (>= 3.4.0)}
\item[Suggests]\AsIs{testthat (>= 2.1.0), covr, pkgdown, tibble}
\item[Imports]\AsIs{readxl (>= 1.0.0), rvest (>= 0.3.2), tools (>= 3.1.0),
lubridate (>= 1.7.0), purrr (>= 0.2.5), readr (>= 1.0.0),
rstudioapi (>= 0.2), xml2 (>= 1.2.0), httr, jsonlite, magrittr}
\item[NeedsCompilation]\AsIs{no}
\item[Author]\AsIs{Ellis Hughes [aut, cre],
Jon Harmon [ctb],
Thomas Mock [ctb],
R4DS Online Learning Community [dtc]}
\item[Maintainer]\AsIs{Ellis Hughes }\email{ellishughes@live.com}\AsIs{}
\end{description}
\Rdcontents{\R{} topics documented:}
\inputencoding{utf8}
\HeaderA{available}{Listing all available TidyTuesdays}{available}
\aliasA{tt\_available}{available}{tt.Rul.available}
\aliasA{tt\_datasets}{available}{tt.Rul.datasets}
%
\begin{Description}\relax
The TidyTuesday project is a constantly growing repository of data sets.
Knowing what type of data is available for each week requires going to the
source. However, one of the hallmarks of 'tidytuesdayR' is that you never
have to leave your R console. These functions were
created to help maintain this philosophy.
\end{Description}
%
\begin{Usage}
\begin{verbatim}
tt_available(auth = github_pat())

tt_datasets(year, auth = github_pat())
\end{verbatim}
\end{Usage}
%
\begin{Arguments}
\begin{ldescription}
\item[\code{auth}] github Personal Access Token. See PAT section for
more information

\item[\code{year}] numeric entry representing the year of tidytuesday you want the
list of datasets for. Leave empty for most recent year.
\end{ldescription}
\end{Arguments}
%
\begin{Details}\relax
To find out the available datasets for a specific year, the user
can use the function `tt\_datasets()`. This function will either populate the
Viewer or print to console all the available data sets and the week/date
they are associated with.

To get the whole list of all the data sets ever released by TidyTuesday, the
function `tt\_available()` was created. This function will either populate the
Viewer or print to console all the available data sets ever made for
TidyTuesday.
\end{Details}
%
\begin{Value}
`tt\_available()` returns a 'tt\_dataset\_table\_list', which is a
list of 'tt\_dataset\_table'. This class has special printing methods to show
the available data sets.

`tt\_datasets()` returns a 'tt\_dataset\_table' object. This class has
special printing methods to show the available datasets for the year.
\end{Value}
%
\begin{Section}{PAT}


A Github PAT is a Personal Access Token. This allows for signed queries to
the github api, and increases the limit on the number of requests allowed
from 60 to 5000. Follow instructions at
<https://happygitwithr.com/github-pat.html> to set your PAT.
\end{Section}
%
\begin{Examples}
\begin{ExampleCode}
# check to make sure there are requests still available
if(rate_limit_check(silent = TRUE) > 10){
 ## show data available from 2018
 tt_datasets(2018)

 ## show all data available ever
 tt_available()
}

\end{ExampleCode}
\end{Examples}
\inputencoding{utf8}
\HeaderA{Available\_Printing}{Printing Utilities for Listing Available Datasets}{Available.Rul.Printing}
\aliasA{print.tt\_dataset\_table}{Available\_Printing}{print.tt.Rul.dataset.Rul.table}
\aliasA{print.tt\_dataset\_table\_list}{Available\_Printing}{print.tt.Rul.dataset.Rul.table.Rul.list}
%
\begin{Description}\relax
printing utilities for showing the available datasets for a specific year or
all time
\end{Description}
%
\begin{Usage}
\begin{verbatim}
## S3 method for class 'tt_dataset_table'
print(x, ..., is_interactive = interactive())

## S3 method for class 'tt_dataset_table_list'
print(x, ..., is_interactive = interactive())
\end{verbatim}
\end{Usage}
%
\begin{Arguments}
\begin{ldescription}
\item[\code{x}] an object used to select a method.

\item[\code{...}] further arguments passed to or from other methods.

\item[\code{is\_interactive}] is the console interactive?
\end{ldescription}
\end{Arguments}
%
\begin{Value}
used for side effects to show the available datasets for the year or for all time.
\end{Value}
%
\begin{Examples}
\begin{ExampleCode}
# check to make sure there are requests still available
if(rate_limit_check(silent = TRUE) > 10){

 available_datasets_2018 <- tt_datasets(2018)
 print(available_datasets_2018)

 all_available_datasets <- tt_available()
 print(all_available_datasets)

}
\end{ExampleCode}
\end{Examples}
\inputencoding{utf8}
\HeaderA{github\_pat}{Return the local user's GitHub Personal Access Token}{github.Rul.pat}
%
\begin{Description}\relax
Extract the GitHub Personal Access Token (PAT) from the system environment
for authenticated requests.
\end{Description}
%
\begin{Usage}
\begin{verbatim}
github_pat(quiet = TRUE)
\end{verbatim}
\end{Usage}
%
\begin{Arguments}
\begin{ldescription}
\item[\code{quiet}] Should this be loud? default TRUE.
\end{ldescription}
\end{Arguments}
%
\begin{Value}
a character vector that is your Personal Access Token, or NULL
\end{Value}
%
\begin{Section}{PAT}


A Github 'PAT' is a Personal Access Token. This allows for signed queries to
the github api, and increases the limit on the number of requests allowed
from 60 to 5000. Follow instructions from
<https://happygitwithr.com/github-pat.html> to set the PAT.
\end{Section}
%
\begin{Examples}
\begin{ExampleCode}

## if you have a personal access token saved, this will return that value
github_pat()

\end{ExampleCode}
\end{Examples}
\inputencoding{utf8}
\HeaderA{printing}{print methods of the tt objects}{printing}
\aliasA{print.tt}{printing}{print.tt}
\aliasA{print.tt\_data}{printing}{print.tt.Rul.data}
%
\begin{Description}\relax
In tidytuesdayR there are nice print methods for the objects that were used
to download and store the data from the TidyTuesday repo. They will always
print the available datasets/files. If there is a readme available,
it will try to display the tidytuesday readme.
\end{Description}
%
\begin{Usage}
\begin{verbatim}
## S3 method for class 'tt_data'
print(x, ...)

## S3 method for class 'tt'
print(x, ...)
\end{verbatim}
\end{Usage}
%
\begin{Arguments}
\begin{ldescription}
\item[\code{x}] a tt\_data or tt object

\item[\code{...}] further arguments passed to or from other methods.
\end{ldescription}
\end{Arguments}
%
\begin{Value}
used to show readme and list names of available datasets

used to show available datasets for the tidytuesday
\end{Value}
%
\begin{Examples}
\begin{ExampleCode}

## Not run:

tt <- tt_load_gh("2019-01-15")
print(tt)

tt_data <- tt_download(tt, files = "All")
print(tt_data)


## End(Not run)
\end{ExampleCode}
\end{Examples}
\inputencoding{utf8}
\HeaderA{rate\_limit\_check}{Get Rate limit left for GitHub Calls}{rate.Rul.limit.Rul.check}
%
\begin{Description}\relax
The GitHub API limits the number of requests that can be sent within an hour.
This function returns the stored rate limits that are remaining.
\end{Description}
%
\begin{Usage}
\begin{verbatim}
rate_limit_check(n = 10, quiet = TRUE, silent = FALSE)
\end{verbatim}
\end{Usage}
%
\begin{Arguments}
\begin{ldescription}
\item[\code{n}] number of requests that triggers a warning indicating the user is
close to the limit

\item[\code{quiet}] should the only an error be thrown when the rate limit is zero?

\item[\code{silent}] should no warnings or errors be thrown and only the value
returned?
\end{ldescription}
\end{Arguments}
%
\begin{Value}
return the number of calls are remaining as a numeric values
\end{Value}
%
\begin{Examples}
\begin{ExampleCode}

rate_limit_check(silent = TRUE)

\end{ExampleCode}
\end{Examples}
\inputencoding{utf8}
\HeaderA{readme}{Readme HTML maker and Viewer}{readme}
%
\begin{Description}\relax
Readme HTML maker and Viewer
\end{Description}
%
\begin{Usage}
\begin{verbatim}
readme(tt)
\end{verbatim}
\end{Usage}
%
\begin{Arguments}
\begin{ldescription}
\item[\code{tt}] tt\_data object for printing
\end{ldescription}
\end{Arguments}
%
\begin{Value}
Does not return anything. Used to show readme of the downloaded
tidytuesday dataset in the Viewer.
\end{Value}
%
\begin{Examples}
\begin{ExampleCode}
## Not run:
tt_output <- tt_load_gh("2019-01-15")
readme(tt_output)

## End(Not run)
\end{ExampleCode}
\end{Examples}
\inputencoding{utf8}
\HeaderA{tt\_download}{download tt data Download all or specific files identified in the tt dataset}{tt.Rul.download}
%
\begin{Description}\relax
download tt data

Download all or specific files identified in the tt dataset
\end{Description}
%
\begin{Usage}
\begin{verbatim}
tt_download(tt, files = c("All"), ..., branch = "master", auth = github_pat())
\end{verbatim}
\end{Usage}
%
\begin{Arguments}
\begin{ldescription}
\item[\code{tt}] string representation of the date of data to pull, in YYYY-MM-dd
format, or just numeric entry for year

\item[\code{files}] List the file names to download. Default to asking.

\item[\code{...}] pass methods to the parsing functions. These will be passed to
ALL files, so be careful.

\item[\code{branch}] which branch to be downloading data from. Default and always
should be "master".

\item[\code{auth}] github Personal Access Token. See PAT section for more
information
\end{ldescription}
\end{Arguments}
%
\begin{Value}
list of tibbles of the files downloaded.
\end{Value}
%
\begin{Section}{PAT}


A Github PAT is a personal Access Token. This allows for signed queries to
the github api, and increases the limit on the number of requests allowed
from 60 to 5000. Follow instructions at
<https://happygitwithr.com/github-pat.html> to set the PAT.
\end{Section}
%
\begin{Examples}
\begin{ExampleCode}
## Not run:
tt_output <- tt_load_gh("2019-01-15")
agencies <- tt_download(tt_output, files = "agencies.csv")

## End(Not run)
\end{ExampleCode}
\end{Examples}
\inputencoding{utf8}
\HeaderA{tt\_download\_file}{Reads in TidyTuesday datasets from Github repo}{tt.Rul.download.Rul.file}
%
\begin{Description}\relax
Reads in the actual data from the TidyTuesday github
\end{Description}
%
\begin{Usage}
\begin{verbatim}
tt_download_file(tt, x, ..., auth = github_pat())
\end{verbatim}
\end{Usage}
%
\begin{Arguments}
\begin{ldescription}
\item[\code{tt}] tt\_gh object from tt\_load\_gh function

\item[\code{x}] index or name of data object to read in. string or int

\item[\code{...}] pass methods to the parsing functions. These will be passed to
ALL files, so be careful.

\item[\code{auth}] github Personal Access Token. See PAT section for more
information
\end{ldescription}
\end{Arguments}
%
\begin{Value}
tibble containing the contents of the file downloaded from git
\end{Value}
%
\begin{Section}{PAT}


A Github PAT is a personal Access Token. This allows for signed queries to
the github api, and increases the limit on the number of requests allowed
from 60 to 5000. Follow instructions at
<https://happygitwithr.com/github-pat.html> to set the PAT.
\end{Section}
%
\begin{Examples}
\begin{ExampleCode}
## Not run:
tt_gh <- tt_load_gh("2019-01-15")

agencies <- tt_download_file(tt_gh, 1)
launches <- tt_download_file(tt_gh, "launches.csv")

## End(Not run)
\end{ExampleCode}
\end{Examples}
\inputencoding{utf8}
\HeaderA{tt\_load}{Load TidyTuesday data from Github}{tt.Rul.load}
%
\begin{Description}\relax
Load TidyTuesday data from Github
\end{Description}
%
\begin{Usage}
\begin{verbatim}
tt_load(x, week, download_files = "All", ..., auth = github_pat())
\end{verbatim}
\end{Usage}
%
\begin{Arguments}
\begin{ldescription}
\item[\code{x}] string representation of the date of data to pull, in YYYY-MM-dd
format, or just numeric entry for year

\item[\code{week}] left empty unless x is a numeric year entry, in which case the
week of interest should be entered

\item[\code{download\_files}] which files to download from repo. defaults and
assumes "All" for the week.

\item[\code{...}] pass methods to the parsing functions. These will be passed to
ALL files, so be careful.

\item[\code{auth}] github Personal Access Token. See PAT section for more
information
\end{ldescription}
\end{Arguments}
%
\begin{Value}
tt\_data object, which contains data that can be accessed via `\$`,
and the readme for the weeks tidytuesday through printing the object or
calling `readme()`
\end{Value}
%
\begin{Section}{PAT}

A Github PAT is a personal Access Token. This allows for signed queries to
the github api, and increases the limit on the number of requests allowed
from 60 to 5000. Follow instructions from
<https://happygitwithr.com/github-pat.html> to set the PAT.
\end{Section}
%
\begin{Examples}
\begin{ExampleCode}

# check to make sure there are requests still available
if(rate_limit_check(silent = TRUE) > 10){

tt_output <- tt_load("2019-01-15")
tt_output
agencies <- tt_output$agencies

}

\end{ExampleCode}
\end{Examples}
\inputencoding{utf8}
\HeaderA{tt\_load\_gh}{Load TidyTuesday data from Github}{tt.Rul.load.Rul.gh}
%
\begin{Description}\relax
Pulls the Readme and URLs of the data from the TidyTuesday
github folder based on the date provided
\end{Description}
%
\begin{Usage}
\begin{verbatim}
tt_load_gh(x, week, auth = github_pat())
\end{verbatim}
\end{Usage}
%
\begin{Arguments}
\begin{ldescription}
\item[\code{x}] string representation of the date of data to pull, in
YYYY-MM-dd format, or just numeric entry for year

\item[\code{week}] left empty unless x is a numeric year entry, in which case the
week of interest should be entered

\item[\code{auth}] github Personal Access Token. See PAT section for more
information
\end{ldescription}
\end{Arguments}
%
\begin{Value}
a 'tt' object. This contains the files available for the week,
readme html, and the date of the tidytuesday.
\end{Value}
%
\begin{Section}{PAT}


A Github PAT is a personal Access Token. This allows for signed queries to
the github api, and increases the limit on the number of requests allowed
from 60 to 5000. Follow instructions from
<https://happygitwithr.com/github-pat.html> to set the PAT.
\end{Section}
%
\begin{Examples}
\begin{ExampleCode}
# check to make sure there are requests still available
if(rate_limit_check(silent = TRUE) > 10){
 tt_gh <- tt_load_gh("2019-01-15")
 readme(tt_gh)
}

\end{ExampleCode}
\end{Examples}
\printindex{}
\end{document}
